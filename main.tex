\documentclass[12pt]{article}
\usepackage[utf8]{inputenc}

\usepackage{hyperref}

\title{Rapport stage}
\author{Max Cohen}
\date{September 2020}

\begin{document}

\begin{titlepage}
    \maketitle
\end{titlepage}

\tableofcontents

\section{Introduction}

\subsection{Context}
The building industry in France makes for over 40\% of the total energy consumption, as well as almost a quarter of greenhouse emissions, as stated by the Loi Grenelle\footnote{\href{https://www.legifrance.gouv.fr/loda/id/JORFTEXT000020949548/2020-09-21/}{Loi Grenelle I, Article 3}} (August 2009), and thus stands as the main industry to tackle when aiming at immediate energy savings. The Fench State acted on this statement by setting a 38\% reduction objective in the consumption of the building industry for 2020\footnote{\href{https://www.legifrance.gouv.fr/loda/id/JORFTEXT000020949548/2020-09-21/}{Loi Grenelle I, Article 5}}, through the renovation of 400 000 apartments per year. Six years later, the amount was raised to 500 000 per year, by the LTECV \footnote{\href{https://www.legifrance.gouv.fr/jorf/id/JORFTEXT000031044385/}{Loi relative à la Transition Energétique pour la Croissance Verte, Article 3}}.
\footnote{\href{https://laffairedusiecle.net/argumentaire-memoire-complementaire/}{Memoire complémentaire de l'Affaire du Siècle, 2019}}.

\subsection{Oze-Energies}

Oze-Energies is a French company specialized in real estate management. Through innovative and durable methods, it aims at improving air quality and confort, while simultaneous reducing energy consumption, without requiring any site work.

The premise of buidling management at Oze is the ability to produce tailored roadmap for managing a building, based on a precise understanding of its behavior, combined with prior knowledge in thermal physics as well as meteo forcasts. While environnemental sensors are integrated in the building using LoRa (a secured and public network), thermics experts known as Energy Managers aim at reaching the best compromise between indoor comfort and energy consumption, while also improving air quality (\ensuremath{\mathrm{CO_2}} levels, humidity, etc.). These roadmaps average in a 25\% reduction in consumption, in just a few weeks : this is Oze-Energies' main product, OPTIMZEN®.

OPTIMZEN® targets tertiary buidlings, whose occupation behavior are usually well understood (workday hours, no occupation during the weekend, no heavy machinery creating heat beside computers and lighting systems), as well as residential buidlings. The analysis of the building, as well as regular roadmaps for the buidling maintainer are delivered for a recuring subscription, usually largely covered by the energy savings generated.

In order to accuratly predict the best management settings for a buidling, a machine learning model is trained to best approximate its behavior, using sensor data collected. Since a wide variety of building behavior can encountered, due to the number of heat, cold or electricity providers for instance, this calibration step becomes more and more important as Oze-Energies aquires new clients.

Oze-Energies was created in 2014, and currently covers over 3 millions $m^2$ over 500 buidlings, \cite{}. During the epidemy of the COVID-19, air quality improvement becomes a pillar of any indoor sanitary measure. Oze-Energies is able to use years of experience in air quality management, in addition to being able to alert on any potential risk, to help creating a safe workplace.

\end{document}