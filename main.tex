\documentclass[12pt]{article}
\usepackage[utf8]{inputenc}
\usepackage{amsmath}
\usepackage{amsthm}
\usepackage{amssymb}
\usepackage{amsfonts}
\usepackage{dsfont}
\usepackage{enumerate}
\usepackage{booktabs}
\usepackage{graphicx}
\usepackage[breaklinks]{hyperref}
\usepackage[hyphenbreaks]{breakurl}
\emergencystretch 2em%
\hypersetup{
    colorlinks=true,
    linkcolor=blue,
    urlcolor=magenta,
}
\graphicspath{ {./images/} }

\newcommand{\eqsp}{\,}

\title{Analysis of metamodel performance for dynamic systems and data}
\author{Max Cohen}
\date{September 2020}

\begin{document}

\begin{titlepage}

    \maketitle

    \begin{center}
        \includegraphics[width=0.2\textwidth]{IPP_Endos_TelecomSudParis_RVB.png}
        \includegraphics[width=0.2\textwidth]{oze_logo.png}    
    \end{center}

\end{titlepage}

\tableofcontents

\section{Abstract}
\subsection{Abstract}
I worked for six month at Oze-Energies, a company specialized in reducing energy consumption in office buildings. I  was  able  to  practice  and  improve  my skills in deep learning, data analysis and software engineering. The following report details the objectives and challenges of my internship, as well as my modest contributions at improving the energy optimization strategy, by using neural network metamodels. Section ~\ref{sec:company} briefly present the company Oze-Energies and the context in which it was developed. Section ~\ref{sec:objectives} develops the objectives for this internship from a technical and personal point of vue. Section~\ref{sec:models} provides all the deep learning architectures used in this report to build a metamodel and describes the data and variables used in our metamodel. Section~\ref{sec:calib:optim} illustrates the performance of our metamodel in the calibration and optimization process of a real building. The numerical experiments illustrate how this metamodel ensures a significant gain in energy in comparison to the considered alternatives.

\subsection{Resumé}
J'ai travaillé pendant six mois chez Oze-Energies, une entreprise spécialisée dans la réduction de la consommation énergétique des bâtiments du tertiaire. J'ai pu y développer mes compétences en deep learning, analyse de donnée et développement logiciel. Ce rapport détaille les objectifs et challenges que j'ai rencontré durant mon stage, ainsi que ma contribution à l'amélioration de la stratégie d'optimisation énergétique, grâce à l'utilisation de méta modèles issus de réseaux de neurones. La section ~\ref{sec:company} présente brièvement l'entreprise Oze-Energies, ainsi que le contexte dans lequel elle s'est développée. Dans la section ~\ref{sec:objectives}, je développe mes objectifs techniques et personnels pour ce stage. La section~\ref{sec:models} décrit précisément les architectures utilisées pour développer le méta modèle, ainsi que les données et variables utilisées. La section~\ref{sec:calib:optim} illustre les performances de notre méta modèle sur les tâches de calibration et optimisation pour un bâtiment réel. Les simulations numériques soulignent le gain significatif en énergie du méta modèle comparé aux alternatives considérées.


\section{Thanks}
I would like to thank the entire team of Oze-Energies for welcoming me, and in particular Maurice Charbit who didn't hesitate to spend countless hours to explain me the responsibility of the R\&D department, the every mechanics of the existing solution, and the ins and outs of working in building management.

I am particularly thankful to Sylvain LeCorff, without whom this very internship could not have happened, and who kept guiding me through my exploration of Oze-Energies and the vast field of statistics.

A special thank to the image department ARTEMIS, who accepted to suffer the long brainstorming sessions completely unrelated to anything visual. 

\newpage

\section{Company}
\label{sec:company}

\subsection{Context}
In 2009, the building industry accounted for over 40\% (see Figure~\ref{fig:consumption_repartition}) of the total French energy consumption, as well as almost a quarter of greenhouse emissions (Loi Grenelle\footnote{\href{https://www.legifrance.gouv.fr/loda/id/JORFTEXT000020949548/2020-09-21/}{Loi Grenelle I, Article 3}}). In the ecological context where energy waste can no longer be ignored, bridging the energy efficiency gap of the building industry would be the first step in reducing the State impact on the environnement\footnote{\href{http://temis.documentation.developpement-durable.gouv.fr/docs/Temis/0067/Temis-0067836/18854.pdf}{Plan d'action national en faveur des énergies renouvelables}}. This translated as a 38\% reduction objective in the consumption of the building industry for 2020, through the renovation of 400 000 apartments per year\footnote{\href{https://www.legifrance.gouv.fr/loda/id/JORFTEXT000020949548/2020-09-21/}{Loi Grenelle I, Article 5}}.

Amid the anticipation of the Paris Agreement, that would eventually require from each endorsing country to communicate their own yearly greenhouse emissions\footnote{\href{https://unfccc.int/files/meetings/paris_nov_2015/application/pdf/paris_agreement_english_.pdf}{Paris Agreement}}, the government issued a new law regarding the ecological transition: the LTECV. It's contribution toward building energy efficiency mainly revolves around improving and speeding up the renovation effort, raising the previous objective to 500 000 apartments per year\footnote{\href{https://www.legifrance.gouv.fr/jorf/id/JORFTEXT000031044385/}{Loi relative à la Transition Energétique pour la Croissance Verte, Article 3}}.

However, despite setting higher and higher objectives toward renovating the French building park, the government still falls short in terms of results, as stated by the Ademe (Agency for the environnement and energy)\footnote{\href{https://www.lefigaro.fr/conjoncture/accord-de-paris-pourquoi-les-pays-ne-sont-pas-a-la-hauteur-de-leurs-engagements-20190419}{Hervé Lefebvre, head of the Climat departiement of the Ademe, 2019}}. According to the National Low-Carbon Strategy (SNBC), the average number of yearly renovation only reaches 370 000 for the period 2015-2030\footnote{\href{https://www.ecologie.gouv.fr/strategie-nationale-bas-carbone-snbc}{Stratégie Nationale Bas-Carbone (SNBC)}}.

\begin{figure}
    \caption{French consumption repartition in 2017}
    \centering
    \includegraphics[width=0.7\textwidth]{energie_consommation_secteurs_2018.png}
    \label{fig:consumption_repartition}
\end{figure}

\subsection{Oze-Energies}

Oze-Energies is a French company specialized in real estate management. Through innovative and durable methods, it aims at improving air quality and comfort, while simultaneous reducing energy consumption, without requiring any site work.

The premise of building management at Oze is the ability to produce tailored road map for managing a building, based on a precise understanding of its behavior, combined with prior knowledge in thermal physics. Environmental sensors are integrated in the building using LoRa (a secured and public network), providing thermic experts known as Energy Managers with real time data of the building. They aim at reaching the best compromise between indoor comfort and energy consumption, while improving air quality (\ensuremath{\mathrm{CO_2}} levels, humidity, etc.) as well. These road maps average in a 25\% reduction in consumption, in just a few weeks : this is Oze-Energies' main product, OPTIMZEN®.

OPTIMZEN® targets tertiary buildings, whose occupation behavior are usually well understood (workday hours, no occupation during the weekend, no heavy machinery creating heat beside computers and lighting systems), as well as residential buildings. The analysis of the building, as well as regular road maps for the building maintainer are delivered for a recurring subscription, usually largely covered by the energy savings generated.

In order to accurately predict the best management settings for a building, the impact of various policy are simulated and compared using a numerical simulators. This expert software, TRNSYS, is calibrated to match the real building by a machine learning procedure, using sensor data collected (see Figure ~\ref{fig:building_model}). Since a wide variety of building behavior can be encountered, due to the number of heat, cold or electricity providers for instance, this calibration step becomes more and more important as Oze-Energies acquires new clients.

Oze-Energies was created in 2014, and currently covers over 3 millions $m^2$ over 500 buildings. During the COVID-19 epidemic, air quality improvement becomes a pillar of any indoor sanitary measure. Oze-Energies is able to use years of experience in air quality management, in addition to being able to alert on any potential risk, to help creating a safe workplace.

\begin{figure}
    \centering
    \includegraphics[width=0.48\textwidth]{bat.jpg}
    \includegraphics[width=0.48\textwidth]{batschem.jpg}
    \caption{Oze-Energies model buildings based on a schematic view of the building and a simulator for transient systems.}
    \label{fig:building_model}
\end{figure}{}

\subsection{Optimizing comfort and energy consumption}
During my internship, I joined the Research and Development team at Oze-Energies, that focuses on using TRNSYS to automatically generate management road maps for buildings.

TRNSYS is an energy simulation software, that is able to simulate the state of a building (interior temperature, consumption) from a precise description of the building (see appendix), weather data and a set of building management settings. Because the software simulates theoretical heat exchanges, and our knowledge of the geometry of the building is often limited, calibration is a mandatory step if we wish to approximate the behavior of a building. In other words, before being able to predict how a building will react to various heating or AC settings, we have to produce an estimation of the characteristics of the building: thermal capacity, volume of airflow exchanges, length of the isolation, window surface, etc.

Once calibrated, we can tune the management settings (comfort and reduced temperature for the HVAC system, starting and stopping schedules, etc.) in order to reduce consumption as mush as possible, while improving comfort. This is a multi objective optimization problem, as improving comfort usually results in higher costs in AC or heating consumption, but we are able to achieve a compromise that usually greatly improves on the current building situation.

Both of the calibration and optimization steps require running an iterative algorithm that can take hundred of thousands of calls to TRNSYS before converging. Considering that a single TRNSYS simulation involves solving complex numerical problems, resulting in computation times no less than 2 seconds, it is plain to see how running these algorithms all the way through convergence is not achievable in practice.

I joined the R\&D team of Oze-Energies in order to speed up the computation time, by using a metamodel of the original TRNSYS software. Training such a model can be quite a time expensive task, but the gain during the calibration and optimization tasks easily makes up for it. The next improvement for the R\&D department after my internship consists in grouping similarly behaving buildings, in order to use a common metamodel for each. With this objective in mind, the initial cost of training a metamodel drops further.

\section{Objectives}
\label{sec:objectives}

\subsection{Technical objectives}
During my internship, we proposed a new end-to-end methodology to optimize the energy performance and the comfort, air quality and hygiene of large buildings. A metamodel based on a Transformer network was introduced and trained using a dataset sampled with a simulation program (TRNSYS). Then, a few physical parameters and the building management system settings of this metamodel are calibrated using the CMA-ES optimization algorithm and real data obtained from sensors. Finally, the optimal settings to minimize the energy loads while maintaining a target thermal comfort and air quality are obtained using a multi-objective optimization procedure. The numerical experiments illustrate how this metamodel ensures a significant gain in energy efficiency while being computationally much more appealing than traditional numerical simulator models, requiring a huge number of physical parameters to be estimated.

\subsection{Expectations for Oze-Energies}
Currently, Energy Managers model each building by a few characteristics variables (size of the isolation, number of occupants, usage, etc.). This model is then sent to the R\&D team who will run the calibration and optimization step, only to send back the results to the Energy Managers, who are responsible for issuing road maps. This back and forth slows down the global process considerably, and should be, in term, replaced by a solution to be used by the Energy Managers. Because their expertise doesn't lie in development or computer science, it is important that this solution be usable without any prior knowledge. From a technical point of vue, this also justifies the development of an end-to-end methodology on my end.

\subsection{Personal formation}
This internship is also an opportunity for me to deepen my knowledge in statistics, optimization and deep learning. All through these six month, my personal formation is always an underlying objective, that translates itself in compiling reviews of optimization and building management research, learning how results in hidden state models can be applied to neural networks, or improving the quality of my code.


\section{Related works}
Global energy demand for heating, ventilation and air-conditioning  in commercial or public buildings has been increasing rapidly for the past few decades along with population and economic growth. This rising demand is at the root of the complex problem of simultaneously ensuring a better environmental impact, as higher consumption of fossil fuels implies higher greenhouse gas emissions, while maintaining a satisfactory comfort in buildings (air and indoor temperature quality). In that respect, building designing and management have to integrate thermal performance and comfort criteria, and to assess the environmental consequences of any chosen policy. This makes the analysis of building energy performance a challenging multi-criteria problem, as detailed for instance in \cite{Bre2016ResidentialBD}. We set the focus on analyzing and optimizing cooling, heating and air conditioning loads by tuning the building management system in given buildings without costly, invasive or time consuming renovation works. The aim is to provide the optimal building management settings, governing Heating, Ventilation and Air-Conditioning (HVAC) and Air Handling Units (AHU), in order to improve thermal comfort, energy loads as well as environmental impact. This objective is decomposed into three steps: (i) provide a model to predict future energy loads and temperature in a building based on the HVAC system and the weather forecast, (ii) calibrate the parameters of this model based on real data obtained in real time in each building and (iii) optimize the HVAC equipment to minimize the total energy load in future periods while maintaining a given thermal comfort.

The first category of approaches to model the energy performance of a building are based on physical equations that describe heat transfer between the building and its environment. Thanks to their increasing reliability, simulation based methods such as EnergyPlus, TRNSYS or DOE-2  are commonly used to simulate the system behavior based on a schematic view of the building. EnergyPlus was used for instance in \cite{shabunko2018energyplus} to build three types of typical designs and to benchmark the energy performance of 400 residential buildings. In \cite{zhao2016occupant}, the authors proposed a predictive control framework based on Matlab and EnergyPlus in order to optimize energy consumptions while meeting the individual thermal comfort preference. In these papers, a schematic building is used in the simulation program and considered as a baseline for energy loads. %This demonstrates the capability of such approaches in producing benchmarks, and conditions for the improvements of energy performance in buildings. 
These approaches rely on a huge number of parameters, such as window to wall ratio, window leakage, or wall construction. Instead of costly campaigns to measure these parameters, that would have to be reiterated for each new building, they may be estimated using an automatic calibration procedure by minimizing a cost function which associates, with each set of parameters, the discrepancy between the true energy loads and temperatures, and the simulated ones, see \cite{Coakley2014ARO, Corff2018OPTIMIZINGTC}. As shown in \cite{Nagpal2019AMF}, calibration yields sufficiently accurate results for a variety of different buildings, thus ensuring limited additional costs to generalize a given model. Once calibrated, the optimization task consists in determining a set of building management settings that will result in lower energy consumption, while preserving comfort. Following numerous works such as \cite{Bre2020AnEM}, the multi-objective Non-dominated Sorting Genetic Algorithm-II (NSGA-II), see \cite{Deb2000AFE}, is the most widespread method to solve the optimization task.  However, when no prior knowledge is available on the thousands of specific parameters required to specify each building, calibrating and optimizing such simulation programs is computationally prohibitive, see \cite{Westermann2019SurrogateMF}. This shortcoming is particularly severe in cases where many data are available from numerous wireless sensors installed in a building but no intrusive and resource consuming in-site campaigns are deployed to fix the values of the physical parameters.

% The use of physical simulators is widespread in several fields such as renewable energy and building simulation, but despite their huge potential, due to the numerous nonlinear solvers they are based on, they often lead to excessive computation time in an calibration or optimization framework \cite{}. 

Metamodeling approaches aim at overcoming this computational cost by proposing surrogate models that replace the physical simulator during calibration and optimization tasks. The parameters of such metamodels are estimated during a training phase using simulations conducted by a physical-based model, that aims at exhaustively capturing the building behavior for various building management settings. In \cite{Bre2020AnEM, Reynolds2018AZB}, statistical models are trained on a dataset sampled from EnergyPlus, allowing significant computational savings during optimization. In \cite{Bre2020AnEM}, the authors proposed to combine NSGA-II with an artificial neural network metamodel to obtain a Pareto front of optimal HVAC parameters with the trained metamodel, in order to optimize the consumption of a $83\,\mathrm{m}^2$ house. %A training and validation dataset were sampled independently using the Latin Hypercube Sampling, introduced in \cite{McKay2000ACO}, to ensure that both splits are representative of the input space. 
To fit this dataset, instead of standard statistical models, %such as polynomial regression, multivariate adaptive regression splines or Gaussian processes, 
this paper uses a Feed Forward Neural Network (FFN) as a metamodel. Although they often yield very accurate predictions, these neural networks are not adapted to time series problems, and are usually substituted for there sequential counter parts, such as recurrent or convolutional based approaches. This FFN was only validated with EnergyPlus simulations, and the calibration step was not performed, as no real historic data from the targeted building were discussed. Similarly, \cite{Reynolds2018AZB} proposed a FFN based metamodeling approach to reduce up to 25\% the energy consumption in a small office building. EnergyPlus was used to sample a dataset for various zones of the building, in order to train zone level metamodels. These simulation spread over 24 hours, and approximated the building behavior from January to March. Once again, in the absence of real historic data, no calibration step was implemented. The first optimization method is similar to previous works, and consists in optimizing consumption by feeding NSGA-II each metamodel. In a novel approach, optimization can also be updated every hour with the newly collected data from the building, in hope to avoid the error drift of simulating 24 hours of building behavior without any feedback. The study presented in \cite{Magnier2010MultiobjectiveOO} focused on the optimization of a $210\,\mathrm{m}^2$ two storey house. Despite measured data being available, no automatic calibration step was discussed ; instead, TRNSYS was calibrated by hand to match the real building, resulting in relative errors of 3.7\%, 3.4\%, and 7.3\% for heating, cooling, and fan monthly energy consumption, respectively. A FFN surrogate model was fit on a dataset of 450 samples, before being optimized using once again NSGA-II. %Although a great deal of research has been devoted to optimize energy consumption, in our paper, we propose an end-to-end methodology which aims at overcoming the following issues of existing approaches: (i) (ii).


Based on this light review, we developed an end-to-end approach, from dataset sampling to metamodel calibration and optimization using data obtained from wireless sensors set in a large building. The proposed metamodels involve classical recurrent neural networks and an time series model based on a Transformer architecture (\cite{Vaswani2017AttentionIA}) which has recently proven both an accurate and computing efficient alternative to traditional sequences to sequences models, such as Long Short-Term Memory (LSTM, \cite{Hochreiter1997LongSM}) and Gated Recurrent Unit (GRU, \cite{Cho2014LearningPR}). Transformers combine an encoder-decoder architecture (\cite{Cho2014LearningPR, Bahdanau2014NeuralMT}), allowing the model to learn semantic information from the observations using attention mechanisms (\cite{Parikh2016ADA, Zhu2019AnES}) that could be interpreted as the day to day patterns of our problem. Once the metamodel is trained on a dataset sampled using TRNSYS, all the parameters of a real building and of its Building Management System (BMS) are estimated using real measurements with the Covariance Matrix Adaptation Evolutionary Strategy (CMA-ES) \cite{igel:hansen:roth:2007} which provides a derivative free optimization procedure. A multi-objective methodology to improve energy efficiency and maintain thermal comfort is then implemented by acting only on the BMS. The NSGA-II approach is used to obtain the Pareto optimal parameters. The performance of this metamodel  are compared at each step with the usual FFN alternatives, LSTM, and GRU  metamodels.

The next sections of this report are organized as follows. Section~\ref{sec:models} provides all the deep learning architectures used in our approach to build a metamodel and describes the data and variables used in our metamodel. Section~\ref{sec:calib:optim} illustrates the performance of our metamodel in the calibration and optimization process of a real building. The numerical experiments illustrate how this metamodel ensures a significant gain in energy in comparison to the considered alternatives.

\section{Metamodeling}
\label{sec:models}

\subsection{Notations}
Let $(X_k)_{k\geqslant 0}$ be the state of the building i.e. the inside temperatures and the consumptions of the building management system. The index $k$ denotes time and, in the setting of Oze-Energies sensors, data are collected each hour. The aim of the metamodel introduced here is to provide a numerically efficient solution to predict $(X_k)_{k\geqslant 0}$ from other variables and external observations such as meteorological data. Such a metamodel is described by several sets of input variables. A parameter $\theta_{\mathrm{build}}$ containing all unknown parameters useful for the geometrical description of the buildings (windows area ratio, etc.) and parameters related to heat transfer  (capacitance, airchange infiltration, etc.). Choosing such parameters allows to build a data set and design a metamodel able to mimic various buildings.%, a specific estimation is then performers with real data to find the best parameter associated with a building.
A sequence $(W_k,O_k,I_k)_{k\geqslant 0}$ providing at each hour the building management system variables in $I_k$ (comfort and reduced temperatures for the HVAC), the occupancy $O_k$ (described as a percentage of a given maximum number of people) and in $W_k$ the weather data at time $k$. In this section, we describe how a simulation program may be used to train the metamodel which aims at mimicking the outputs of this simulation program for  various choices of $\theta_{\mathrm{build}}$, $(I_k)_{k\geqslant 0}$, $(O_k)_{k\geqslant 0}$ and of meteorological data $(W_k)_{k\geqslant 0}$. The supplementary material displays a complete list is of the variables contained in $\theta_{\mathrm{build}}$, $(I_k)_{k\geqslant 0}$, $(O_k)_{k\geqslant 0}$ and in $(W_k)_{k\geqslant 0}$ for our numerical experiment.

\subsection{Models}
In most recent works, a great deal of research activities focused on FFN as surrogate models, \cite{Bre2020AnEM, Magnier2010MultiobjectiveOO, Reynolds2018AZB}. Although they may lead to interesting performance during the training phase, these fully connected architectures are not well suited for time series prediction, in particular for long time spans. We ceased this opportunity to explore other approaches that have proven to be more relevant for solving time series tasks in the past few years. Therefore, we decided to evaluate the go-to architectures for time series: a standard LSTM, a bidirectional GRU (BiGRU), a hybrid model mixing both convolutional and GRU layers (ConvGru), and a Feed Forward Network (FFN) as used in previous works.  In addition to those models, a Transformer model which  introduces an attention mechanism to model dependencies is also considered. These models have been implemented using the deep learning framework PyTorch, and can be found on our Github\footnote{https://pytorch.org and https://github.com/maxjcohen/transformer}.

Recurrent Neural Network (RNN) were first introduced as a more suited architecture for dealing with time varying input patterns \cite{Mozer1989AFB}. By replacing buffer based approaches with an updated context state, RNN are able to solve time series problems with short time dependencies, but are lackluster in problems requiring long term memory due to vanishing and exploding gradient \cite{Bengio1994LearningLD}. Long Short Term Memory proposed in \cite{Hochreiter1997LongSM} aim at bridging that gap by enforcing error flow throughout time in the network. Later, \cite{Cho2014LearningPR} modified the LSTM architecture in order to simplify implementation and improve computation times, resulting in a novel model called Gated Recurrent Unit.

In parallel to these advances on recurrent architectures, Convolutional Neural Networks (CNN), rendered popular by \cite{Krizhevsky2012ImageNetCW} for image classification, have been adapted to time series problem. The approaches proposed in \cite{Jzefowicz2016ExploringTL,Kim2016CharacterAwareNL} outperformed traditional Natural Language Processing (NLP) models by replacing the embedding layer with a character-level convolutional layer. Following this idea, \cite{Oord2016WaveNetAG} considerably improved on the speech to text state of the art, by using dilated convolutions, increasing the receptive fields of WaveNet at each layer. One year later, \cite{Oord2017ParallelWF} improved on the existing architecture by introducing Parallel WaveNet, which provided similar performance for a lower computational cost.

Recurrent and convolutional approaches coincide in that temporally close time steps data are matched together. In 2017, \cite{Vaswani2017AttentionIA} proposed an attention based approach to solving NLP tasks that consider the entire input sequence in parallel. The Transformer model is based on a self-attention mechanism (see Figure ~\ref{fig:self_attention}), that computes an attention value for every element of a sequence with respect to all others to model their dependency. This attention mechanism allows to understand at each time step $k$ which input elements are crucial to predicting the new state $X_k$. This makes these networks more interpretable than their most widely-used recurrent counterparts such as LSTM or GRU networks and motivate a keen interest for such approach to predict complex time series.


Transformer differ from sequential architectures in that they compute prediction at each time step in parallel. In our context, we propose to use a Transformer architecture as follows. Let $F_{\theta_{\mathrm{meta}}}$ be the Transformer mapping which computes a prediction  $(\widehat X_{k})_{1\leqslant k\leqslant n}$ of the states $(X_{k})_{1\leqslant k\leqslant n}$:
$$
    (\widehat X_1,\ldots,\widehat X_n) = F_{\theta_{\mathrm{meta}}}(\theta_{\mathrm{build}}, (W_k,O_k,I_k)_{1\leqslant k\leqslant n})\,,
$$
where $\theta_{\mathrm{meta}}$ contains all the unknown parameters specific to the metamodel.
%The first layer of the original Transformer model is a mandatory Embedding layer in NLP tasks, that aims at mapping input words to a meaningful space. 
Following the state of the art in sequence to sequence modeling, the Transformer  adopts an encoder-decoder architecture (see Figure ~\ref{fig:encoder_decoder}). The encoder computes a latent vector from the input data, which is fed to the decoder in order to predict the outputs. These sub-networks are trained jointly and are supposed to foster learning of a meaningful representation of the data. The encoder and decoder consist of a self attention block, responsible for leveraging the relationship between time steps in the sequence, and a feed forward network, which contains the non linearity of the Transformer.

\paragraph{Embedding.}
Similarly to the original embedding layer,  our metamodel is first based on a linear map, that allows setting the dimension $d_{\mathrm{emb}}$ of the latent representation of the inputs.
Let $\Delta >0$ be an  attention window and $k$ be a given time step. For all $k-\Delta \leqslant j\leqslant k+\Delta$, let $U_j = (\theta_{\mathrm{build}}, I_j, W_j, O_j)\in\mathbb{R}^d$ be the vector at time $j$ which stacks all inputs, and $U^{\mathrm{emb}}_j$ the latent vector for the corresponding time step,
$$
    U^{\mathrm{emb}}_j = W_{\mathrm{emb}} \cdot U_j + b_{\mathrm{emb}}\eqsp,
$$
where $W_{\mathrm{emb}}\in\mathbb{R}^{d_{\mathrm{emb}}\times d}$ and $b_{\mathrm{emb}}\in\mathbb{R}^{d_{\mathrm{emb}}}$ are the unknown weight matrix and bias respectively, that are estimated during the training phase.
%and are contained in $\theta_{\mathrm{meta}}$. \textcolor{red}{ }
%It is defined in the original paper as a combination of linear transformations, with a ReLU activation function \cite{}. With parameter $\theta_{\mathrm{att}}$:
%$$\texttt{FFN}_{\theta_{\mathrm{att}}}(z) = W_2 \cdot max(0, W_1 \cdot z + b_1)  + b_2$$
%The encoder and decoder consist of a self attention block, responsible for leveraging the relationship between time steps in the sequence, and a feed forward network (FFN), which contains the non linearity of the Transformer. %It is defined in the original paper as a combination of linear transformations, with a ReLU activation function \cite{}. With parameter $\theta_{\mathrm{att}}$:
%$$\texttt{FFN}_{\theta_{\mathrm{att}}}(z) = W_2 \cdot max(0, W_1 \cdot z + b_1)  + b_2$$

\paragraph{Encoder.}
The encoder block proceeds by computing the query, key and value from $R_j$ for this state with a linear transform:
\begin{equation}
    \label{eq:qkv}
    q_j =  W^{q} U^{\mathrm{emb}}_j\eqsp,\quad \kappa_j =  W^{\kappa} U^{\mathrm{emb}}_j\eqsp,\quad v_j =  W^{v} U^{\mathrm{emb}}_j \eqsp,
\end{equation}
where $W^{q}$, $W^{\kappa}$ and $W^{v}$ are the unknown $r\times d_{\mathrm{emb}}$ matrices (parameters of the metamodel to be estimated, $r$ chosen by the user). Then, let $K_k$ denote the matrix whose columns are $\kappa_j$, $k-\Delta\leqslant j\leqslant k + \Delta$,  and compute for all $k-\Delta\leqslant j\leqslant k+\Delta$,
$$
    s_k^{j} = q_j^TK_k \quad\mbox{and}\quad \pi_k^{j}= \sigma(s_k/\sqrt{r})_j\eqsp,
$$
where $\sigma$ is the softmax function. Finally, self-attention is computed as
\begin{equation}
    \label{eq:z}
    z^{\mathsf{enc}}_k =  \sum_{\ell=k-\Delta}^{k+\Delta}\pi_k^{\ell}v_\ell\eqsp.
\end{equation}
The output of the encoder is then given by a final transform of $z^{\mathsf{enc}}_k$ which is considered as the input of a FFN:
$$
    r^{\mathsf{lat}}_k = \texttt{FFN}_{\theta_{\mathrm{att}}}(z^{\mathsf{enc}}_k)\eqsp,
$$
where $\theta_{\mathrm{att}} = \{W_1,b_1,W_2,b_2\}$ and
$$
    \texttt{FFN}_{\theta_{\mathrm{att}}}(z) = W_2 \cdot max(0, W_1 \cdot z + b_1)  + b_2\,.
$$
In practice, the self attention computation \eqref{eq:z} is replicated $h$ times, each referred to as "head", that are concatenated before being fed to the FFN. Having multiple heads, i.e. computing multiple instances of self attention in parallel, allows the transformer to set attention to multiple aspect of the input sequence at the same time. In a multi-layer Transformer, the output of each layer is used as an input for the next layer before being processed similarly.

\paragraph{Decoder.} The decoder block acts similarly, except for one added attention step where the keys and values are computed from the latent vectors $r^{\mathsf{lat}}_j$, $k-\Delta\leqslant j \leqslant k+\Delta$.
%For all time step $k\leqslant j\leqslant k+\Delta$, a self attention layer is performed using $(U_\ell)_{k\leqslant \ell\leqslant j}$ to produce a first output as \eqref{eq:z}. Then, the output of this layer is used as an input to compute a hidden representation as in \eqref{eq:qkv}. 
This produces a vector $z^{\mathsf{enc}}_k$ as in \eqref{eq:z} which is a mixture of the values associated with the latent vectors. This mixture is fed to a FFN to produce $\widehat X_k$.  The parameters to train are therefore $W_{\mathrm{emb}}$, $b_{\mathrm{emb}}$, $\theta_{\mathrm{att}}$ and $W^{q}$, $W^{\kappa}$ and $W^{v}$.

\begin{figure}
    \centering
    \includegraphics[width=0.8\textwidth]{illustration_attentionmap.jpg}
    \caption{The resulting attention map for the Transformer after inferring a month of data. We can identify day/night cycles, as well as week/weekend.}
    \label{fig:self_attention}
\end{figure}{}

\begin{figure}
    \centering
    \includegraphics[width=0.5\textwidth]{illustration_encoder_decoder.png}
    \caption{The encoder computes a latent vector from the input data, which is fed to the decoder in order to predict the outputs. These sub-networks are trained jointly and are supposed to encourage the Transformer to learn a meaningful representation of the data.}
    \label{fig:encoder_decoder}
\end{figure}{}

\subsection{Training and validation}
\label{sec:training}
The first step consists in sampling a dataset with TRNSYS to learn the metamodel and  defining ranges for each input parameters in $\theta_{\mathrm{build}}$, $(I_k)_{k\geqslant 0}$ and $(O_k)_{k\geqslant 0}$ with the help of energy managers, such as highest and lowest scheduled temperature, or the most early and late hour of arrival of occupants, see the supplementary material for a complete list of these ranges. In addition, real weather data $(W_k)_{k\geqslant 0}$ acquired between May and December 2019 where used to obtain a dataset consistent with the real building. As discussed in the previous section, some related papers use Latin Hypercube sampling, introduced in \cite{McKay2000ACO}, to form their dataset. In our numerical experiments, we chose instead a uniform sampling method over the ranges of each variable. This allows us to easily split the dataset into $k$-folds, which will be useful for the validation step discussed in the next section.

During this step, daily values defined in the appendix are converted to a time series whose value changes with every day. This way, there are 38 variables in the input vector at each time step: 19 variables from $\theta_{\mathrm{build}}$, 7 from $W_k$, 1 from $O_k$ and 11 from $I_k$. A total of 38000 training examples were sampled, an example being a week i.e. 168 hours.
During the training phase, the parameters of each metamodel described in Section~\ref{sec:models} are estimated based on this dataset (called  $\theta_{\mathrm{meta}}$ in the detailed case of the Transformer approach). The metamodels compared in this section are defined with a latent dimension of $d_{emb} = 64$ and a total of $N=8$ layers. These values were obtained through a grid search, see the supplementary material for additional information. Other hyper parameters, such as learning rate dropout, number of epochs or batch size, were chosen empirically.

%In order to chose the most efficient network, we ran a Grid Search for the Transformer model to set the hyperparameters, allowing us to decide on various dimensions such as the latent vector dimension or the number of layers. For instance, we trained four networks with different latent dimensions, and settled on a compromise between computation time and accuracy by setting the latent dimension to 64 (see the supplementary material for additional information). Other Hyper Parameters, learning rate, dropout, number of epochs to train or batch size among others, were chosen empirically. 

During training, for each example, we use a loss function defined by Energy Management experts, consisting of a combination between mean squared consumption and temperature errors:
\begin{align*}
    \Delta_T^{\theta_{\mathrm{meta}}} = \left(\frac{1}{N}\sum_{k=1}^N (\widehat T_k^{\theta_{\mathrm{meta}}} - T_k)^2\right)^{1/2}\quad\mathrm{and} & \quad
    \Delta_Q^{\theta_{\mathrm{meta}}} = \left(\frac{1}{N}\sum_{k=1}^N (\widehat Q_k^{\theta_{\mathrm{meta}}} - Q_k)^2\right)^{1/2}\,,                                                                              \\
    \mathrm{loss}(\theta_{\mathrm{meta}}) = \alpha\log(1 + \Delta_T^{\theta_{\mathrm{meta}}})                                                       & + \beta \cdot \log(1 + \Delta_Q^{\theta_{\mathrm{meta}}})\,,
\end{align*}
where $N$ is the number of data in each example, $T_k$ and $Q_k$ are the ground truth at time $k$, and $\widehat T_k^{\theta_{\mathrm{meta}}}$ and $\widehat Q_k^{\theta_{\mathrm{meta}}}$ are the predictions given by the metamodel with the current value $\theta_{\mathrm{meta}}$ of the metamodel for temperature and consumption respectively. In this experiments below, we chose $\alpha=1$ and $\beta=0.3$. We chose the Adam optimizer \cite{Kingma2015AdamAM} ; all simulations were computed on a single 1080TI GPU card. Table~\ref{table:train} displays the mean values and standard deviations of the loss function on the validation dataset after training. The table also displays the mean squared error $\mathrm{MSE}_T$ (resp. $\mathrm{MSE}_Q$) on the temperatures (resp. consumptions) only, and these metrics computed only during occupation time $\mathrm{MSE_T^{occ}}$ and $\mathrm{MSE_Q^{occ}}$.  In addition, the coefficients of determination (rescaled  mean squared errors relative to the
target data) of the temperatures $R^2_T$ and consumptions $R^2_Q$ are given. These coefficients of determination are computed with the Python function {\em sklearn.metrics.r2\_score}.

%and of mean squared errors of the temperature and cooling consumptions  In addition, Figure~\ref{} illustrates the performance of the metamodel on two randomly chosen test samples.


\begin{table}
    \caption{Metrics (means and standard deviations) of the metamodels on the validation dataset. The best mean values are displayed in bold (the lowest losses and mean squared errors and the coefficient of determination closest to 1).}
    \label{table:train}
    \centering
    \begin{tabular}{*7c}\toprule
                                                       & Transformer                  & BiGRU              & LSTM               & ConvGru            & FFN           \\
        \toprule
        $\mathrm{Loss}$ \;\;\;\;\;($\times10^{-4}$)    & $\textbf{1.13\eqsp (0.746)}$ & $1.43\eqsp (1.06)$ & $13.8\eqsp (4.55)$ & $2.78\eqsp (1.77)$ & $61.1 (27.4)$ \\
        %\hline
        $\mathrm{MSE_T}$ \;\;($\times10^{-5}$)         & $\textbf{3.86\eqsp (4.53)}$  & $4.28\eqsp (5.18)$ & $7.32\eqsp (7.75)$ & $9.37\eqsp (11.5)$ & $178 (205)$   \\
        %\hline
        $\mathrm{MSE_Q}$ \;\;($\times10^{-4}$)         & $\textbf{2.47\eqsp (2.30)}$  & $3.34\eqsp (2.98)$ & $43.7\eqsp (14.7)$ & $6.16\eqsp (4.30)$ & $146 (54.2)$  \\
        %\hline
        $\mathrm{MSE_T^{occ}}$ ($\times10^{-5}$)       & $\textbf{1.08\eqsp (1.32)}$  & $1.18\eqsp (1.54)$ & $2.02\eqsp (2.52)$ & $2.77\eqsp (3.37)$ & $51.2 (64.1)$ \\
        %\hline
        $\mathrm{MSE_Q^{occ}}$  ($\times10^{-4}$)      & $\textbf{1.06\eqsp (1.29)}$  & $1.21\eqsp (1.92)$ & $3.61\eqsp (2.93)$ & $2.28\eqsp (2.35)$ & $43.2 (25.1)$ \\
        %\hline
        \emph{$R^2_T$} \;\;\;\;\;\;\;($\times10^{-3}$) & $\textbf{996\eqsp (0.832)}$  & $996\eqsp (1.40)$  & $992\eqsp (1.64)$  & $990\eqsp (2.10)$  & $829 (43.2)$  \\
        %\hline
        \emph{$R^2_Q$} \;\;\;\;\;\;\;($\times10^{-3}$) & $\textbf{760\eqsp (240)}$    & $657\eqsp (593)$   & $559\eqsp (473)$   & $707 (268)$        & $-738 (3080)$ \\
        %\begin{tabular}{*8c}\toprule
        % & \emph{Loss}  & \emph{$MSE_T$} & \emph{$MSE_Q$}  & \emph{$MSE_T^{occ}$} & \emph{$MSE_Q^{occ}$} & \emph{$R^2_T$} & \emph{$R^2_Q$} \\\toprule
        %Transformer & $1.53\cdot10^{-4}$ & $6.56\cdot10^{-6}$ & $1.82\cdot10^{-4}$ & $1.66\cdot10^{-6}$ & $8.04\cdot10^{-5}$ & 0.995 & 0.953  \\
        %BiGRU & $2.64\cdot10^{-4}$ & $1.63\cdot10^{-5}$	& $2.16\cdot10^{-4}$ & $4.67\cdot10^{-6}$ & $8.53\cdot10^{-5}$ & $0.989$ & $0.951$ \\
        \bottomrule
    \end{tabular}
\end{table}

\section{Energy Optimization in a real building}
\label{sec:calib:optim}
The experiments lead us to analyze the performance of the metamodel trained in Section~\ref{sec:training} focused on the optimization of a $28733\, m^2$ building located in the Parisian region. The total building is represented by a single thermal zone including  5 vertical walls with respective following areas $3521\, m^2$, $2692\, m^2$, $3257\, m^2$, $599\, m^2$ and $16329\, m^2$, a horizontal roof and a horizontal ground. Based on a commonly used rule, it is assumed that $2/3$ of the full area is occupied by people. Assuming that each occupant requires $12\, m^2$, this allows to set the initial values for the number of occupants and the number of PCs (set to 1.2 times this value) in the building during occupancy hours. These values are assumed to be known and fixed and used to sample the training dataset.

\subsection{Calibration}
During the training phase, metamodel parameters are estimated by minimizing the loss function on the simulated dataset which corresponds to various choices of $\theta_{\mathrm{build}}$, $(I_k,O_k,W_k)_{k\geqslant 0}$, associated with building behaviors $(X_k)_{k\geqslant 0}$. This metamodel has been trained on a dataset containing only simulated data,  ignoring real building related noise and measurement errors. Additionally, both the BEM and our surrogate model take as input a number of variables, such as window to wall ratio, window leakage, or wall construction, that cannot be properly identified for each building. By comparing the metamodel predictions to real historic data during the calibration phase, we search for a set of building related parameters that best match reality.

During this step, the weights $\theta_{\mathrm{meta}}$ of the metamodel are frozen, meaning that we no longer back propagate the error, nor do we update each weight matrix of the neural network. Using the coefficient of determination as a cost function, we can compute, for each given set of input parameters $\theta_{\mathrm{build}}$, $(I_k,O_k,W_k)_{k\geqslant 0}$, the difference between estimated and real historical data. Because this is a non differentiable problem, the cost function cannot be minimized using the same algorithm as in the training step; instead we use the CMA evolution strategy (CMA-ES, \cite{Hansen2016TheCE}), an evolutionary algorithm adapted to derivative free non-convex optimization problems in continuous domain. It is implemented by the author of the paper in the \texttt{pycma} library\footnote{https://github.com/CMA-ES/pycma}.

Calibration was run until convergence for the metamodel, and for a maximum of 8 hours for the original BEM (TRNSYS). We can see the advantage of going through the training of a metamodel when comparing a calibration for both TRNSYS and the metamodel, as we are now able to reach lower costs in a much shorter time frame. This is confirmed by Table~\ref{tab:calib} which displays the Mean squared error for the temperatures and heating consumption after calibration using TRNSYS and the Transformer-based metamodel, for two different weeks shown in Figure~\ref{fig:calib}.

\begin{table}
    \caption{Metrics after calibration for two weeks, beginning the 4th and the 30th of November 2019. Calibration run for 500 epochs (resp. 2500 epochs) for the metamodel (resp. for TRNSYS).}
    \label{tab:calib}
    \centering
    \begin{tabular}{*9c}    \toprule
                     & $\mathrm{MSE_T}$   & $\mathrm{MSE_Q}$ & $\mathrm{MSE_T^{occ}}$ & $\mathrm{MSE_Q^{occ}}$ & \emph{$R^2_T$} & \emph{$R^2_Q$} & time (h)   \\\toprule
        {\bf Week 1} &                    &                  &                        &                        &                &                &          & \\
        TRNSYS       & $1.04\cdot10^{-1}$ & $4967$           & $3.31\cdot10^{-2}$     & $1434$                 & 0.644          & 0.848          & 2          \\
        Metamodel    & $1.62\cdot10^{-2}$ & $3241$           & $4.71\cdot10^{-3}$     & $477$                  & 0.945          & 0.901          & 2          \\
        \midrule
        {\bf Week 2} &                    &                  &                        &                        &                &                &          & \\
        TRNSYS       & $2.66\cdot10^{-1}$ & $16067$          & $6.58\cdot10^{-2}$     & $6782$                 & 0.592          & 0.761          & 2          \\
        Metamodel    & $1.42\cdot10^{-1}$ & $10493$          & $6.55\cdot10^{-2}$     & $5162$                 & 0.782          & 0.844          & 2          \\
        \bottomrule
    \end{tabular}
\end{table}

\begin{figure}
    \centering
    \includegraphics[width=\textwidth]{calib_week1_private.png}
    \includegraphics[width=\textwidth]{calib_week1_t_int.png}
    \includegraphics[width=\textwidth]{calib_week2_private.png}
    \includegraphics[width=\textwidth]{calib_week2_t_int.png}
    \caption{Consumption and temperature simulations after calibration, for both the metamodel and TRNSYS, for week 1 (top) and week 2 (bottom). Green bars indicate occupation periods.}% Table \ref{tab:calib} contains metrics for this calibration step.}
    \label{fig:calib}
\end{figure}

\subsection{Optimization}
Once the metamodel is calibrated, we can use it as an accurate simulator for how the building will react to changes in its usage. After a successful calibration, all building related variables contained in $\theta_{\mathrm{build}}$ are correctly estimated. The parameters $I_k$ associated with the HVAC system can be optimized for a given set of weather data $W_k$. The optimization tasks consists in finding a set a usage related parameters that reduce consumption while keeping the same level of comfort. Optimizing energy consumption requires minimizing two conflicting objectives, making it impossible to find a solution that optimize both objectives simultaneously. Instead, we search for optimal compromises between energy consumption and comfort, in the form of a Pareto front. Indeed, for any such optimal compromise, we can always get a higher level of comfort, for the price of a higher consumption. The consumption criteria is the energy load during the week ; the comfort criteria is the gap between indoor temperature and a constant reference temperature $T^*$:
%Regarding consumptions, the chosen optimization loss is the energy load during the week; and regarding comfort, it is the mean squared error between the predicted indoor temperature and a constant reference temperature $T^*$:
\begin{align*}
    \Delta^{\mathrm{opt}}_T = \frac{1}{N^{\mathrm{opt}}_{Occ}}\left(\sum_{k=1}^{N^{\mathrm{opt}}} \mathds{1}_{k\in \mathrm{Occ}}(\widehat T_k - T^*)^2\right)^{1/2}\quad\mathrm{and} & \quad
    \Delta^{\mathrm{opt}}_Q = \frac{1}{N^{\mathrm{opt}}} \sum_{k=1}^{N^{\mathrm{opt}}} \widehat Q_k\,,
\end{align*}
%\begin{align*}
%    \mathds{1}_{k\in Occ}
%\end{align*}
where $T^*=22.5^{\circ}C$, $N^{\mathrm{opt}}$ is the number of hours to be considered in the optimization process and $\mathrm{Occ}$ is a subset of daytime hours specifying at which hours the target temperature has to be reached in the building.
%
%\begin{align}
%\Delta_T &= \sqrt{\int (y_{est}^T - t^*)^2}\eqsp, \\
%\Delta_Q &= \int y_{est}^Q\eqsp.
%\end{align}{}
Following recent works in building energy optimization, we search for a set of optimal parameters using NSGA-II (\cite{Deb2000AFE}), another evolutionary algorithm, but adapted to multi objective problems. An implementation can be found in the \texttt{Pygmo} library\footnote{\href{https://esa.github.io/pygmo2/}{https://esa.github.io/pygmo2/}}. In the absence of a stopping condition, we simply run the optimization for a set 3000 epochs (2 hours). The result can be viewed as a Pareto front which is given in Figure \ref{fig:pareto} for the second week used in the calibration process. As observed during calibration, this process can take a colossal number of epochs before achieving satisfactory results, once again justifying the use of a much faster metamodel. The time series of consumption and temperatures associated with the BMS parameters selected in Figure~\ref{fig:pareto} are given in Figure~\ref{fig:timeseriesafteroptim}.

\begin{figure}
    \centering
    \includegraphics[width=0.8\textwidth]{pareto_week2.png}
    \caption{Pareto front after optimization for the second week. We select the point of closest equivalent comfort, corresponding to a 9.31\% reduction in consumption.}
    \label{fig:pareto}
\end{figure}


\begin{figure}
    \centering
    \includegraphics[width=\textwidth]{optim_week2_private.png}
    \includegraphics[width=\textwidth]{optim_week2_t_int.png}
    \caption{Consumption and temperature simulations after optimization (metamodel) for the second week. Green bars indicate occupation periods.} %Consumption is stopped during the night, leading to lower temperature outside occupation periods.}
    \label{fig:timeseriesafteroptim}
\end{figure}

\section{Conclusion}
%Optimizing building energy consumption is a challenging task, due to the need for carefully integrated sensors in the building, the multi objective nature of the problem, as well as the computational cost of calibration or optimization procedures. For this reason, most related works focus one a single aspect of the problem.
\subsection{Results}
We proposed an end-to-end metamodeling methodology to optimize building energy loads and to reduce computational costs. The proposed metamodel ensures compatibility between simulations and real building observations through a calibration step. We experimented with various deep learning architectures more suited to recurrent problems than Feed Forward Networks. Results show that a wide variety of models display encouraging results on our sampled dataset, while largely outperforming FFN. During optimization, we chose to maintain the same level of comfort as the historical data, in order to have as little impact as possible on the working environment. Compared to calibrated simulations, we were able to reduce consumption significantly.

This metamodel approach significantly improves on the existing optimization method of the R\&D department of Oze-Energies, mainly by reducing computation time. Our experiments show that a model can be calibrated and optimized in no more than a few hours, allowing for many optimization strategies to be assessed in the span of a single day. Indeed, if the metamodel is currently only being used by the R\&D, it's underlying purpose is eventually to become a supporting tool for the Energy Managers in establishing road maps, and thus requires to run simulations in a limited amount of time.

\subsection{Perspectives}
There are many approaches to reducing the computation time of the R\&D pipeline, but the metamodel actually offers more than just a computationally cheap alternative to TRNSYS. Considering the encouraging results in calibrating and optimizing for a single building, we are now considering using a single neural network to model a group of similarly behaving buildings. By training on such a cluster, we further reduce the initial cost of the metamodel compared to the original TRNSYS, but we also gain deeper knowledge of the most influential inter-building variables, and their impact.

The metamodel can also be interpreted as a classic hidden markov model, allowing for various opportunity of improvement. For instance, by adding model noise to the existing architecture, we can evaluate the uncertainty of the model, and provide confidence intervals along each prediction. If this provides a good validation of the model from a statistical point of vue, it also contributes to the edition of more plausible road maps for Oze-Energies' clients.

\subsection{Personal assessment}
This internship was first and foremost a unique work experience. During these past six months, I worked along a team of experts dedicated to tackle on the biggest environmental challenges of our generation. If there are more and more professional opportunities for deep learning engineers, rarely do they focus on such essential tasks, and in such an impactful manner. Being able to put to good use my experience in statistics and deep learning to good use was truly a delightful experience.

I was also able to dive into the building management field, and better understand how construction methods, building usage and thermic laws impact our day to day consumption. Thanks to the thermic engineers at Oze-Energies who took the time to explain these processes to me, and convey their motivation, I was able to better grasp the ins and outs of building management.

From a scientific perspective, these six month allowed me discover new aspects of my field, and develop my knowledge in statistics and modelling. They also encouraged me to focus and explore a precise topic of research, an experience which will be without a doubt valuable in the continuation of my career as a PhD student.

\bibliography{references}
\bibliographystyle{apalike}
\clearpage
\appendix

\section{Ranges}
\begin{table}
	\centering
	\begin{tabular}{@{}llll@{}}
		Variable                             & Minimum & Maximum & Step \\ \midrule
		airchange\_infiltration\_vol\_per\_h & 0.1     & 0.5     & 0.1  \\
		capacitance\_kJ\_perdegreK\_perm3    & 50      & 300     & 10   \\
		power\_VCV\_kW\_heat                 & 0       & 1000    & 100  \\
		power\_VCV\_kW\_clim                 & 0       & 1000    & 100  \\
		nb\_occupants                        & 1000    & 2000    & 200  \\
		nb\_PCs                              & 1000    & 2000    & 200  \\
		percent\_light\_night                & 0       & 70      & 10   \\
		percent\_PCs\_night                  & 0       & 70      & 10   \\
		facade\_1\_thickness\_2              & 0.05    & 0.15    & 0.05 \\
		facade\_2\_thickness\_2              & 0.05    & 0.15    & 0.05 \\
		facade\_3\_thickness\_2              & 0.05    & 0.15    & 0.05 \\
		facade\_4\_thickness\_2              & 0.05    & 0.15    & 0.05 \\
		roof\_1\_thickness\_3                & 0.05    & 0.15    & 0.05 \\
		facade\_1\_window\_area\_percent     & 40      & 50      & 5    \\
		facade\_2\_window\_area\_percent     & 40      & 50      & 5    \\
		facade\_3\_window\_area\_percent     & 40      & 50      & 5    \\
		facade\_4\_window\_area\_percent     & 40      & 50      & 5    \\
		\bottomrule                                                     \\
	\end{tabular}
	\caption{$\theta\_buidling$ ranges.}
	\label{tab:theta-building}

\end{table}

\begin{table}
	\centering
	\begin{tabular}{@{}llll@{}}
		Variable                & Minimum & Maximum & Step \\ \midrule
		start\_clim\_day        & 7       & 9       & 1    \\
		end\_clim\_day          & 18      & 20      & 1    \\
		t\_clim\_red\_day       & 24      & 30      & 0.5  \\
		t\_clim\_conf\_day      & 20      & 24      & 0.5  \\
		start\_heat\_day        & 6       & 8       & 1    \\
		end\_heat\_day          & 17      & 19      & 1    \\
		t\_heat\_red\_day       & 17      & 22      & 0.5  \\
		t\_heat\_conf\_day      & 22      & 24      & 0.5  \\
		start\_ventilation\_day & 7       & 9       & 1    \\
		end\_ventilation\_day   & 18      & 20      & 1    \\
		t\_ventilation\_day     & 18      & 26      & 0.5  \\
		vol\_ventilation\_day   & 0.7     & 1.7     & 0.3  \\
		\bottomrule                                        \\
	\end{tabular}
	\caption{$I_k$ ranges. Each parameter can hold a different value for each day of the week. For ease of reading, we replaced them by a single line, as the ranges are the same for every day.}
	\label{tab:Ik}

\end{table}

\begin{table}
	\centering
	\begin{tabular}{@{}llll@{}}
		Variable                     & Minimum & Maximum & Step \\ \midrule
		start\_occupation\_monday    & 7       & 9       & 1    \\
		start\_occupation\_tuesday   & 7       & 9       & 1    \\
		start\_occupation\_wednesday & 7       & 9       & 1    \\
		start\_occupation\_thursday  & 7       & 9       & 1    \\
		start\_occupation\_friday    & 7       & 9       & 1    \\
		end\_occupation\_monday      & 17      & 20      & 1    \\
		end\_occupation\_tuesday     & 17      & 20      & 1    \\
		end\_occupation\_wednesday   & 17      & 20      & 1    \\
		end\_occupation\_thursday    & 17      & 20      & 1    \\
		end\_occupation\_friday      & 17      & 20      & 1    \\
		\bottomrule                                             \\
	\end{tabular}
	\caption{$O_k$ ranges.}
	\label{tab:Ok}
\end{table}

\begin{table}
	\centering
	\begin{tabular}{*2c}
		Variable & Description                    \\ \midrule
		DNI      & Direct Normal Irradiance       \\
		IBEAM\_H & Direct Horizontal Irradiance   \\
		IBEAM\_N & Direct Normal Irradiance       \\
		IDIFF\_H & Diffuse Horizontal Irradiation \\
		IGLOB\_H & Global Horizontal Irradiance   \\
		RHUM     & Humidity                       \\
		TAMB     & Outdoor temperature            \\
		\bottomrule                               \\
	\end{tabular}
	\caption{Weather data as contained in $W_k$.}
	\label{tab:Wk}
\end{table}

\section{Labels}
\begin{table}
	\centering
	\begin{tabular}{@{}llll@{}}
		variable        & description                                     \\ \midrule
		Q\_AC\_OFFICE   & AC consumption                                  \\
		Q\_HEAT\_OFFICE & Heat consumption                                \\
		Q\_PEOPLE       & Heating power from human activities             \\
		Q\_EQP          & Equipment (computers, elevators, fridges, etc.) \\
		Q\_LIGHT        & Consumption of lights                           \\
		Q\_AHU\_C       & Consumption of AHU when cooling outside air     \\
		Q\_AHU\_H       & Consumption of AHU when heating outside air     \\
		T\_INT\_OFFICE  & Indoor temperature                              \\
		\bottomrule                                                       \\
	\end{tabular}
	\caption{BEM's output variables at each time step.}
	\label{tab:output_variables}

\end{table}

% \begin{table}[]
%     \centering
%     \begin{tabular}{@{}llll@{}}
%     variable \\ \midrule
% 	ac\_t\_conf \\ 
% 	ac\_t\_red \\ 
% 	ac\_mask \\ 
% 	heat\_t\_conf \\ 
% 	heat\_t\_red \\ 
% 	heat\_mask \\ 
% 	ventilation\_t \\ 
% 	ventilation\_vol \\ 
% 	ventilation\_mask \\ 
% 	pc\_on\_mask \\ 
% 	lights\_on\_mask \\ 
%  \bottomrule
%     \end{tabular}
%     \caption{$I_k$}

% \end{table}

% \begin{table}[]
%     \centering
%     \begin{tabular}{@{}llll@{}}
%     variable \\ \midrule
% 	occupancy \\ 
%     \bottomrule \\
%     \end{tabular}
%     \caption{$O_k$}

% \end{table}

% \begin{table}[]
%     \centering
%     \begin{tabular}{@{}llll@{}}
%     variable \\ \midrule
% 	airchange\_infiltration\_vol\_per\_h \\ 
% 	capacitance\_kJ\_perdegreK\_perm3 \\ 
% 	power\_VCV\_kW\_heat \\ 
% 	power\_VCV\_kW\_clim \\ 
% 	nb\_occupants \\ 
% 	nb\_PCs \\ 
% 	facade\_1\_thickness\_2 \\ 
% 	facade\_1\_window\_area\_percent \\ 
% 	facade\_2\_thickness\_2 \\ 
% 	facade\_2\_window\_area\_percent \\ 
% 	facade\_3\_thickness\_2 \\ 
% 	facade\_3\_window\_area\_percent \\ 
% 	facade\_4\_thickness\_2 \\ 
% 	facade\_4\_window\_area\_percent \\ 
% 	roof\_1\_thickness\_3 \\ 
% 	init\_day \\ 
% 	init\_month \\ 
% 	init\_year \\ 
% 	initial\_temperature \\ 
%  \bottomrule
%     \end{tabular}
%     \caption{$\theta_{{build}}$}

% \end{table}


\section{Dataset distribution}
\begin{figure}
    \centering
    \includegraphics[width=0.9\textwidth]{dataset_distribution_Z_168h.png}
    \caption{Dataset distribution for input variables: occupancy, AC usage and outdoor temperature.}
    \label{fig:dataset_distribution_Z}
\end{figure}

\begin{figure}
    \centering
    \includegraphics[width=0.7\textwidth]{dataset_distribution_X_168h.png}
    \caption{Dataset distribution for output variables.}
    \label{fig:dataset_distribution_X}
\end{figure}


\end{document}